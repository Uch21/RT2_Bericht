%--------------------------------------------------------------------
%
%Mustervorlage fuer eine Aufgabe
%
%--------------------------------------------------------------------
%
%Ueberschreiben der automatisch erzeugten Aufgabennummer
%Die folgende Aufgabennummer ergibt sich aus dem Stand des
%Z?hlers + 1
%\setcounter{chapter}{0}
%
\chapter{Beschreibung der �bertragungsfunktionen}\label{ex:uebertrag}
%
%Teilaufgabe 1
%
\section{Regelstrecke}\label{sec:regelstrecke}
%
Ausgehend von dem Gleichgewicht der Kr�fte $F_Z + F_R - m*g - m*a = 0$\\
Daraus ergibt sich folgende Differentialgleichung:
\begin{equation}
	\frac{32\mu_0 l^*}{\rho v D^{*^2}} \rho v^2 A + 6\pi \mu_0 vr - mg = m \ddot{x}
\end{equation}
Nach K�rzen ergibt sich:
\begin{equation}
\frac{32\mu_0}{D^*} v A + 6\pi\mu_0 vr - mg = m\ddot{x}
\end{equation}
Nach der Laplacetransformation erh�lt man:
\begin{equation}
\frac{32\mu_0}{D^*} v A + 6\pi\mu_0 vr - mg = m *\text{s}^2 x
\end{equation}
Umgestellt nach dem Weg:
\begin{equation}
\frac{1}{\text{s}^2} * (\frac{32\mu_0}{m D^*} v A + \frac{6\pi\mu_0 vr}{m} - g) = x
\end{equation}
%
Das Ergebnis wird mittels Einheiten gepr�ft:
\begin{equation}
\frac{\mu_0 v A}{mD^*} = \frac{[Pa*s * m * m^2]}{[kg * m]} = \frac{[N * m^4 * s]}{[N * s^3 * m^3]} = \frac{[m]}{[s^2]}
\end{equation}
\begin{equation}
	\frac{\mu_0 v r}{m} = \frac{[Pa*s*m*M]}{[kg*s]} = \frac{[N*m^3*s]}{[N*s^3*m^2]} = \frac{[m]}{[s^2]}
\end{equation}
\begin{equation}
	g = \frac{[m]}{[s^2]}
\end{equation}
Die Beschleunigung zweimal integriert ergibt den Weg.
%--------------------------------------------------------------------
%
%
\newpage
\section{Abstandssensor}
%
Durch Linearisierung im Arbeitspunkt x = 45 cm ergibt sich folgende �bertragungsfunktion:
\begin{equation}
	U(x) = -1.25x + 1.25
\end{equation}
\section{L�fter}
Durch Linearisierung im Arbeitspunkt \textit{U} = 5V ergibt folgende �bertragungsfunktion:
\begin{equation}
U(x) = -0.018x + 0.13
\end{equation}
%Alle bisherigen Bilder einf?gen und einen Seitenumbruch erzwingen
\clearpage